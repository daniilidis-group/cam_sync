\section{Auto exposure}
\newcommand{\Bdes}{B_{\mathrm{des}}}
\newcommand{\Bcurr}{B_{\mathrm{curr}}}
\newcommand{\Bthresh}{B_{\mathrm{thresh}}}
\newcommand{\Scurr}{S_{\mathrm{curr}}}
\newcommand{\Smax}{S_{\mathrm{max}}}
\newcommand{\Gmax}{G_{\mathrm{max}}}
\newcommand{\Gcurr}{G_{\mathrm{curr}}}

\label{sec:auto_exposure}
The FLIR/PointGrey computer vision cameras used for our experiments
have a built-in auto exposure algorithm. However, auto exposure mode
is incompatible with hardware based shutter triggering, requiring
the implementation of an external software based exposure control.

For completeness, we describe the simple exposure control scheme
\cite{Liang2007} that 
is implemented in CamSync. The control algorithm adjusts both the
shutter time $S$ and the gain $G$. Gain is only used when the target
brightness cannot be reached by increasing the shutter time.

When a frame is received for camera $i$, the intensity values of the
pixels at every 32nd row and every 32nd  column of the image (in raw
Bayer format) are averaged. At 1920x1200 resolution, this equates to
2250 pixels that are sampled, and takes a negligible amount of CPU
time. The so computed image brightness $B$ is in the range of 0 to
255. If $B$ deviates by more than $\Bthresh$ (typically set to 10) from the
brightness target $\Bdes$ (typically about 90), then either shutter or
gain settings are updated.

Let $\Scurr$ denote the currently effective shutter time.
If $\Scurr$ is less than the configured maximum
shutter time $\Smax$ then the new commanded shutter speed is:
\begin{equation}
 S = \mathrm{min}(\Scurr \frac{\Bdes}{\Bcurr},\Smax)\ .
\end{equation}

If $\Scurr = \Smax$, gain control is used to reach the desired
brightness:
\begin{equation}
 G = \mathrm{min}(\Gcurr + 10 \log_{10}(\Bdes/\Bcurr),\Gmax)\ .
\end{equation}
Note that here the gain is expressed in decibels (db).

The commands to update gain or shutter speed are sent immediately
after the frame has been processed. However a few frames will elapse
until the driver's commands reach the camera. For
this reason the CamSync implementation will wait for a configurable
number of frames before it begins to evaluate the image brightness
again.
